

Finally free slip boundary conditions have been implemented, but only at the 
surface, and only with the method of Lagrange Multipliers (\stone~\ref{f151}
taught us that it works as well as the other method).  
{\color{red} change}

\begin{eqnarray}
\K_e \cdot \vec{\cal V} + \G_e \cdot \vec{\cal P} &=& \vec{f}  \\
\G_e \cdot \vec{\cal V} &=& \vec{0}
\end{eqnarray}
We multiply the first line by the rotation matrix ${\bm R}$:
\begin{eqnarray}
{\bm R} \cdot \K_e \cdot \vec{\cal V} +{\bm R} \cdot \G_e \cdot \vec{\cal P} &=&{\bm R} \cdot \vec{f}  \\
\G_e \cdot \vec{\cal V} &=& \vec{0}
\end{eqnarray}
and then introduce the identity matrix ${\bm I}={\bm R}^T\cdot {\bm R}$ before the velocity vector:
\begin{eqnarray}
{\bm R} \cdot \K_e \cdot {\bm R}^T\cdot {\bm R} \cdot  \vec{\cal V} +{\bm R} \cdot \G_e \cdot \vec{\cal P} &=&{\bm R} \cdot \vec{f}  \\
\G_e \cdot  {\bm R}^T\cdot {\bm R} \cdot  \vec{\cal V} &=& \vec{0}
\end{eqnarray}
The second line can also be written
\begin{eqnarray}
({\bm R} \cdot \K_e \cdot {\bm R}^T) \cdot ({\bm R} \cdot  \vec{\cal V}) + ({\bm R} \cdot \G_e) \cdot \vec{\cal P} &=&{\bm R} \cdot \vec{f}  \\
( {\bm R} \cdot \G_e)^T \cdot  ({\bm R} \cdot  \vec{\cal V}) &=& \vec{0}
\end{eqnarray}
which translates at the elemental level into
\begin{lstlisting}
K_el=RotMat.dot(K_el.dot(RotMat.T))
f_el=RotMat.dot(f_el)
G_el=RotMat.dot(G_el)
\end{lstlisting}
Note that the matrix $\K_e$ is $(m*ndofV)\times (m*ndofV)$ in size, and so is the matrix ${\bm R}$.

After boundary conditions are imposed, the system is rotated back:

\begin{lstlisting}
K_el=RotMat.T.dot(K_el.dot(RotMat))
f_el=RotMat.T.dot(f_el)
G_el=RotMat.T.dot(G_el)
\end{lstlisting}
